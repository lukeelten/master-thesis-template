\documentclass{fiwthesis}

% ========
%  Pakete
% ========

\usepackage{textgreek}           % griechische Buchstaben außerhalb des Math-Mode
\usepackage{amsmath}             % zentrierte Formeln
\usepackage{amssymb}             % erweiterter Formelsatz mathem. Symbole

\usepackage{boldline}            % breitere Linien in Tabellen
\usepackage{booktabs}            % typographisch richtige Tabellen setzen
\usepackage{tabularx}            % Erweiterte Tabellendarstellung
\usepackage{multirow}            % Spalte über mehrere Zeilen oder Spalten ausdehnen
\usepackage{xltabular}           % Zeilenumbrüche in tabularx erlauben

\usepackage{graphicx}            % ermöglicht das Einbinden von Grafiken
\usepackage{subcaption}          % mehrere Bilder in einem Bild
\usepackage{pgfplots}            % Grafiken erzeugen
\usepackage{smartdiagram}        % schnelle und einfache Grafiken
\usetikzlibrary{positioning}     % bessere Ortsbezeichnung
\usetikzlibrary{shapes}          % typische Formen wie Rechtecke, Ellipsen usw. einfach zeichnen
\usetikzlibrary{intersections}   % Schnittpunkt von Geraden adressieren
\usetikzlibrary{angles, quotes}  % einfacheres Zeichnen von Winkeln
\usetikzlibrary{                 % Symbole für Schaltpläne
  circuits.logic.US,
  circuits.logic.IEC,
  circuits.logic.CDH,
  circuits.ee.IEC
}

\usepackage{lipsum}

% ===========
%  Metadaten
% ===========

\thesis{Bachelor-Thesis}
\title{Die Grundlagen der Arithmetik: Eine logisch-mathematische Untersuchung über den Begriff der Zahl}
\author{Gottlob Frege}
\matrnr{123456}
\bdate{08.11.1848}
\bcity{Wismar}
\supervisor{Prof.~Dr.~Erika Mustermann}
\secsupervisor{John Doe}
\keywords{Logik, Mathematik}

% Metadaten in die PDF-Datei schreiben
\makepdfmetadata

% ===============
%  Präambel
% ===============

% PGF Kompatibilitätseinstellung
\pgfplotsset{width=0.95\textwidth,compat=newest}

% % Bibliographie einbinden
\bibliography{quellen}

% Glossar einbinden
\newglossaryentry{nosql}{%
  name = {NoSQL},
  description = {Kurzform für ,,Not Only SQL``; Überbegriff für Datenbanken, die das Konzept relationaler Datenbanken erweitern}
}

\newdualentry{dos}% label
{DoS}% short form
{Denial of Service}% long form
{Ein Denial of Service (im Deutschen: Dienstverweigerung) ist ein Angriffe auf Computer- oder Netzwerksysteme, wobei das Zielsystem durch Überlastung oder durch andere Mittel außer Betrieb gesetzt wird}% description

% Abkürzungen einbinden
%\gls{}         normal zu nutzen (erstes Mal: 'lange Form (kurze Form)'), danach nur 'kurze Form'
%\glspl{}       wie \gls{} nur als Plural
%\acrfull{qrc}  gibt volle Form ('lange Form (kurze Form)') egal wo
%\acrlong{qrc}  gibt lange Form ('lange Form') egal wo
%
%\newacronym{tag}{short}{long}
\newacronym{lamp}{LAMP}{Linux, Apache, MySQL, PHP}
\newacronym{qrc}{QR-Code}{Quick Response Code}


% Symbole einbinden
\newglossaryentry{symb:phi}{
  name=$\phi$,
  description={Ein beliebiger Winkel},
  sort=symbolphi, type=symbolslist
}

\newglossaryentry{symb:e}{
  name=$e$,
  description={Die Eulersche Zahl},
  sort=symbole, type=symbolslist
}


% Glossar- und Abkürzungsverzeichniserstellung
\makeglossaries{}

% Index erzeugen
\makeindex[
  intoc=true,
  title=Index,
  columns=2]{}
\indexsetup{headers={\indexname}{\indexname}}

% ===============
%  Eigene Makros
% ===============

\newcommand*{\code}[1]{\texttt{#1}}

% ===============
%  Beginn Thesis
% ===============

\begin{document}

% ============
%  "Vorspann"
% ============

% Titelseite
\maketitle

% Aufgabenstellung
\maketask{\textbf{\underline{ACHTUNG!}}
\par
Die ausgehändigte Originalaufgabenstellung (und bei jeder Kopie
die entsprechenden Kopie) wird ohne Seitenzahlangabe
eingebunden. Bei deutschsprachigen Aufgabenstellungen wird
der Titel in englischer Sprache wiederholt.
\par
Für die digitale Fassung der Arbeit ist eine Schilderung der Aufgabenstellung aber durchaus sinnvoll und kann an dieser Stelle verfasst werden.
}

% Abstract
\makeabstract{
  Maximal eine halbe Seite.
}{
  English Version.
}

% Inhaltsverzeichnis (Schalter `compact' sorgt für einfachen Zeilenabstand)
\maketoc[compact]

% ==========
%  Textteil
% ==========

% Einleitung
\chapter{Einleitung}
\label{einleitung}

Einleitung in die Arbeit.


% weitere Kapitel hier jeweils einzeln einbinden
\chapter{Beispiele}\label{beispiel}

Beispielkapitel.

\section{Ein Abschnitt}
Beispielabschnitt.

Aufzählungen werden mit der \code{enumerate} Umgebung erstellt:
\begin{enumerate}
  \item{Beispielpunkt A}
  \item{Beispielpunkt B}
  \item{\ldots}
\end{enumerate}

Sollen nur Stichpunkte abgebildet werden, so nimmt man dafür eine \code{itemize} Umgebung:
\begin{itemize}
  \item{Beispielpunkt C}
  \item{Beispielpunkt D}
  \item{\ldots}
\end{itemize}

\subsection{Ein Unterabschnitt}

Beispieltext.

\subsubsection{Ein Unter-Unterabschnitt}

Das ist die niedrigste Ebene.

\section{Tabellen}

Tabelle~\ref{tab:bsp} ist eine Beispieltabelle.
Man beachte die Position der Beschriftung.


\begin{table}[!htbp]
  \caption{Beispieltabelle}\label{tab:bsp}
  \centering{%
    \begin{tabular}{|c|c|}
      \hline
      \textbf{Zeitpunkt (s)} & \textbf{Wert} \\
      \hlineB{3}
      0                      & 0.0           \\
      \hline
      1                      & 0.3           \\
      \hline
      2                      & 0.9           \\
      \hline
    \end{tabular}
  }
\end{table}

Ein wenig aufwendiger ist Tabelle \ref{tab:bsp2}.


\begin{table}[!htbp]
  \caption{Tabelle mit \code{tabularx}, farbigen Zellen und Multicolumn.}
  \label{tab:bsp2}
  \centering {
    \sffamily
    \begin{tabularx}{0.3\textwidth}{|X|l|}
      \hline
      \rowcolor{gray!30!white}\multicolumn{2}{|c|}{\textbf{Schema} EAV} \\
      \hline
      \rowcolor{gray!30!white}\textbf{Spalte} & \textbf{Datentyp}       \\
      \hline
      \underline{id}                          & INTEGER                 \\
      \hline
      entität                                 & VARCHAR                 \\
      \hline
      attribut                                & VARCHAR                 \\
      \hline
      wert                                    & FLOAT                   \\
      \hline
    \end{tabularx}
  }
\end{table}

\section{Grafiken}

\section{Quellcode-Listings}

Minted lässt inline Code wie z.B. \mintinline[style=vs]{c}{print("Hallo, LaTeX!")} zu.

Für Listings können Dateien zum Einbinden angegeben werden (Listing \ref{lst:example-file}).

\begin{listing}[!htbp]
  \inputminted{c}{./anlagen/exampleCode.c}
  \caption{C-Quelltext aus Datei}
  \label{lst:example-file}
\end{listing}

Alternativ kann der Quelltext direkt in eine \code{minted} Umgebung eingefügt werden (Listing \ref{lst:example-intext}).

\begin{listing}[!htbp]
  \begin{minted}{c}
    #include <stdio.h>

    int main(void)
    {
      printf("Hallo nochmal!\n");

      return 0;
    }
  \end{minted}
  \caption{Weiteres Beispiel für C-Quelltext}
  \label{lst:example-intext}
\end{listing}

Beide Beispiele werden im voreingestellten Stil dargestellt.
Das Paket \code{minted} bietet weitere Farbschemata, wie das Beispiel \ref{lst:example-color} zeigt.
In der Präambel des Dokuments kann mit \mintinline{tex}{\setminted{style=...}} der globale Stil der Listings angepasst werden.

\begin{listing}[!htbp]
  \renewcommand\theFancyVerbLine{%
    \rmfamily
    \textcolor[rgb]{0.7,0.7,0.7}{\tiny {\arabic{FancyVerbLine}}}%
  }
  \begin{minted}[style=solarized-dark, bgcolor=solarized@base03]{cpp}
    #include <iostream>

    int main(void)
    {
      std::cout << "Hallo (diesmal in Farbe)!" << std::endl;

      return 0;
    }
  \end{minted}
  \caption{C++ Quelltext im \textit{Solarized} Farbschema}
  \label{lst:example-color}
\end{listing}

Eine Auswahl von bereits definierten Styles ist auf der Webseite von Pygments (\url{https://pygments.org/styles/}) zu finden.

\textbf{Aber Achtung:} Die Zeilennummerierung ist standardmäßig schwarz und kann nur durch das Überschreiben von \textit{\textbackslash theFancyVerbLine} geändert werden.
Dies kann global in der Präambel (siehe \code{renewcommand...} in Listing \ref{lst:example-color}) für alle Listings geschehen oder lokal (ebenfalls Listing \ref{lst:example-color}) in der \code{listing} Umgebung.


\section{Referenzen}
\label{refn}

In Kapitel \ref{beispiel} auf Seite \pageref{refn} finden Sie einige Beispiele dafür, wie Referenzen in \LaTeX~funktionieren.

\subsection{Abkürzungen}

Eine weit verbreitete Architektur für Web-Anwendungen ist der \gls{lamp}-Stack (Beispiel für die Nutzung eines Akronyms).
Wird das gleiche Akronym nochmals verwendet, wird automatisch die Kurzform \gls{lamp}-Stack verwendet.
Pluralformen sind ebenfalls automatisiert möglich, so wird aus dem \gls{qrc} im Plural die \glspl{qrc}.
Außerdem ist es möglich die volle Form, wie beim ersten Benutzen (\acrfull{qrc}), oder nur die ausgeschriebene Form (\acrlong{qrc}) zu wiederholen.

\subsection{Glossar}

MongoDB ist ein Datenbanksystem, das in die Kategorie der \gls{nosql}-Datenbanken fällt (Beispiel für einen Eintrag ins Glossar).
Manchmal wird eine Mischung aus Glossareintrag und Akronym benötigt, zum Beispiel um einen eigentlich geläufigen Fachbegriff wie \gls{dos} zu erklären.

\subsection{Symbolverzeichnis}

\begin{equation}
  \alpha = \frac{1}{e} + sin(\phi)
\end{equation}

Hier die Symbole \gls{symb:phi} und \gls{symb:e}, welche im Symbolverzeichnis erscheinen, um ihre Bedeutung zu erklären.

\subsection{Literatur}

Und natürlich kann auch auf Literatur verwiesen werden.
Alle Quellen werden in diesem Beispiel in die Datei \textit{quellen.bib} geschrieben.
In \citetitle{unterstein12} \cite{unterstein12} geht es beispielsweise um Datenbanken.
Der Artikel von \citeauthor{goldberg91} \cite{goldberg91} ist auch ganz interessant.
Zum Schluss noch eine online Quelle \cite{wave} und eine lange URL \cite{long}, die im \nameref{bib} hoffentlich ordentlich auf mehrere Zeilen aufgeteilt wird.


% Schluss
\chapter{Zusammenfassung und Ausblick}
\label{schluss}

Rückblick, Bewertung, Ausblick über mögliches Fortführen der Arbeit


% =========
%  Anlagen
% =========

\begin{appendices}

  \chapter{Beispielanlage}\label{app:bsp}

Beispieltext.


\end{appendices}

% ===============
%  Verzeichnisse
% ===============

% Verzeichnisse mit einzeiligem Zeilenabstand
\singlespacing

% Literaturverzeichnis
\listofreferences

% Abbildungsverzeichnis einfügen
\listoffigures

% Tabellenverzeichnis einfügen
\listoftables

% Algorithmenverzeichnis einfügen
\listofalgorithms

% % Quelltextverzeichnis einfügen
\listoflistings

% Abkürzungsverzeichnis
\listofacronyms

% Symbolverzeichnis
\listofsymbols

% falls ein anderer Glossar-Stil genutzt wird und die zweite Spalte zu schmal ist:
% \setlength{\glsdescwidth}{0.8\linewidth}

% Glossar einfügen
\printglossary

% Index einfügen
\printindex

% wieder auf 1½-fachen Zeilenabstand umschalten
\normalspacing

% =========================================
%  Selbstständigkeitserklärung, CD, Thesen
% =========================================

% Selbstständigkeitserklärung
\makedeclarationofindependence

% Inhalt der CD; nur für gedruckte Version wichtig
\makecd{\small

\begin{minipage}[t]{0.97\linewidth}
  \dirtree{%
    .1 /\DTcomment{Wurzelverzeichnis}.
    .2 OrdnerA\DTcomment{Ein Ordner auf dem Datenträger}.
    .3 OrdnerB\DTcomment{Ein Unterordner auf dem Datenträger}.
    .3 datei.xyz\DTcomment{Eine Datei}.
    .2 thesis.pdf\DTcomment{PDF-Datei dieser Bachelor-Thesis}.
  }
\end{minipage}

\vspace{2em}

Im Unterverzeichnis \code{tools} des Projekts findet sich das Perl-Skript \code{dirtree.pl}, mit welchem Inhalte für das dirtree-Environment (siehe oberhalb) semiautomatisch erstellt werden können.

Die Nutzung aus der Kommandozeile ist wie folgt:\\
\code{perl dirtree.pl /path/to/top/of/dirtree}

Quelle des Skripts:\\
\url{https://texblog.org/2012/08/07/semi-automatic-directory-tree-in-latex/}
\normalsize}

% Thesen (allerletzte Seite); Thesen bitte immer durch Semikolons trennen
\maketheses{%
  These 1;
  These 2;
  \ldots
}

\end{document}

% =============
%  Ende Thesis
% =============
