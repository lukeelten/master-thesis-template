\small

\begin{minipage}[t]{0.97\linewidth}
  \dirtree{%
    .1 /\DTcomment{Wurzelverzeichnis}.
    .2 OrdnerA\DTcomment{Ein Ordner auf dem Datenträger}.
    .3 OrdnerB\DTcomment{Ein Unterordner auf dem Datenträger}.
    .3 datei.xyz\DTcomment{Eine Datei}.
    .2 thesis.pdf\DTcomment{PDF-Datei dieser Bachelor-Thesis}.
  }
\end{minipage}

\vspace{2em}

Im Unterverzeichnis \code{tools} des Projekts findet sich das Perl-Skript \code{dirtree.pl}, mit welchem Inhalte für das dirtree-Environment (siehe oberhalb) semiautomatisch erstellt werden können.

Die Nutzung aus der Kommandozeile ist wie folgt:\\
\code{perl dirtree.pl /path/to/top/of/dirtree}

Quelle des Skripts:\\
\url{https://texblog.org/2012/08/07/semi-automatic-directory-tree-in-latex/}
\normalsize