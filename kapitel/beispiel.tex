\chapter{Beispiele}\label{beispiel}

Beispielkapitel.

\section{Ein Abschnitt}
Beispielabschnitt.

Aufzählungen werden mit der \code{enumerate} Umgebung erstellt:
\begin{enumerate}
  \item{Beispielpunkt A}
  \item{Beispielpunkt B}
  \item{\ldots}
\end{enumerate}

Sollen nur Stichpunkte abgebildet werden, so nimmt man dafür eine \code{itemize} Umgebung:
\begin{itemize}
  \item{Beispielpunkt C}
  \item{Beispielpunkt D}
  \item{\ldots}
\end{itemize}

\subsection{Ein Unterabschnitt}

Beispieltext.

\subsubsection{Ein Unter-Unterabschnitt}

Das ist die niedrigste Ebene.

\section{Tabellen}

Tabelle~\ref{tab:bsp} ist eine Beispieltabelle.
Man beachte die Position der Beschriftung.


\begin{table}[!htbp]
  \caption{Beispieltabelle}\label{tab:bsp}
  \centering{%
    \begin{tabular}{|c|c|}
      \hline
      \textbf{Zeitpunkt (s)} & \textbf{Wert} \\
      \hlineB{3}
      0                      & 0.0           \\
      \hline
      1                      & 0.3           \\
      \hline
      2                      & 0.9           \\
      \hline
    \end{tabular}
  }
\end{table}

Ein wenig aufwendiger ist Tabelle \ref{tab:bsp2}.


\begin{table}[!htbp]
  \caption{Tabelle mit \code{tabularx}, farbigen Zellen und Multicolumn.}
  \label{tab:bsp2}
  \centering {
    \sffamily
    \begin{tabularx}{0.3\textwidth}{|X|l|}
      \hline
      \rowcolor{gray!30!white}\multicolumn{2}{|c|}{\textbf{Schema} EAV} \\
      \hline
      \rowcolor{gray!30!white}\textbf{Spalte} & \textbf{Datentyp}       \\
      \hline
      \underline{id}                          & INTEGER                 \\
      \hline
      entität                                 & VARCHAR                 \\
      \hline
      attribut                                & VARCHAR                 \\
      \hline
      wert                                    & FLOAT                   \\
      \hline
    \end{tabularx}
  }
\end{table}

\section{Grafiken}

\section{Quellcode-Listings}

Minted lässt inline Code wie z.B. \mintinline[style=vs]{c}{print("Hallo, LaTeX!")} zu.

Für Listings können Dateien zum Einbinden angegeben werden (Listing \ref{lst:example-file}).

\begin{listing}[!htbp]
  \inputminted{c}{./anlagen/exampleCode.c}
  \caption{C-Quelltext aus Datei}
  \label{lst:example-file}
\end{listing}

Alternativ kann der Quelltext direkt in eine \code{minted} Umgebung eingefügt werden (Listing \ref{lst:example-intext}).

\begin{listing}[!htbp]
  \begin{minted}{c}
    #include <stdio.h>

    int main(void)
    {
      printf("Hallo nochmal!\n");

      return 0;
    }
  \end{minted}
  \caption{Weiteres Beispiel für C-Quelltext}
  \label{lst:example-intext}
\end{listing}

Beide Beispiele werden im voreingestellten Stil dargestellt.
Das Paket \code{minted} bietet weitere Farbschemata, wie das Beispiel \ref{lst:example-color} zeigt.
In der Präambel des Dokuments kann mit \mintinline{tex}{\setminted{style=...}} der globale Stil der Listings angepasst werden.

\begin{listing}[!htbp]
  \renewcommand\theFancyVerbLine{%
    \rmfamily
    \textcolor[rgb]{0.7,0.7,0.7}{\tiny {\arabic{FancyVerbLine}}}%
  }
  \begin{minted}[style=solarized-dark, bgcolor=solarized@base03]{cpp}
    #include <iostream>

    int main(void)
    {
      std::cout << "Hallo (diesmal in Farbe)!" << std::endl;

      return 0;
    }
  \end{minted}
  \caption{C++ Quelltext im \textit{Solarized} Farbschema}
  \label{lst:example-color}
\end{listing}

Eine Auswahl von bereits definierten Styles ist auf der Webseite von Pygments (\url{https://pygments.org/styles/}) zu finden.

\textbf{Aber Achtung:} Die Zeilennummerierung ist standardmäßig schwarz und kann nur durch das Überschreiben von \textit{\textbackslash theFancyVerbLine} geändert werden.
Dies kann global in der Präambel (siehe \code{renewcommand...} in Listing \ref{lst:example-color}) für alle Listings geschehen oder lokal (ebenfalls Listing \ref{lst:example-color}) in der \code{listing} Umgebung.


\section{Referenzen}
\label{refn}

In Kapitel \ref{beispiel} auf Seite \pageref{refn} finden Sie einige Beispiele dafür, wie Referenzen in \LaTeX~funktionieren.

\subsection{Abkürzungen}

Eine weit verbreitete Architektur für Web-Anwendungen ist der \gls{lamp}-Stack (Beispiel für die Nutzung eines Akronyms).
Wird das gleiche Akronym nochmals verwendet, wird automatisch die Kurzform \gls{lamp}-Stack verwendet.
Pluralformen sind ebenfalls automatisiert möglich, so wird aus dem \gls{qrc} im Plural die \glspl{qrc}.
Außerdem ist es möglich die volle Form, wie beim ersten Benutzen (\acrfull{qrc}), oder nur die ausgeschriebene Form (\acrlong{qrc}) zu wiederholen.

\subsection{Glossar}

MongoDB ist ein Datenbanksystem, das in die Kategorie der \gls{nosql}-Datenbanken fällt (Beispiel für einen Eintrag ins Glossar).
Manchmal wird eine Mischung aus Glossareintrag und Akronym benötigt, zum Beispiel um einen eigentlich geläufigen Fachbegriff wie \gls{dos} zu erklären.

\subsection{Symbolverzeichnis}

\begin{equation}
  \alpha = \frac{1}{e} + sin(\phi)
\end{equation}

Hier die Symbole \gls{symb:phi} und \gls{symb:e}, welche im Symbolverzeichnis erscheinen, um ihre Bedeutung zu erklären.

\subsection{Literatur}

Und natürlich kann auch auf Literatur verwiesen werden.
Alle Quellen werden in diesem Beispiel in die Datei \textit{quellen.bib} geschrieben.
In \citetitle{unterstein12} \cite{unterstein12} geht es beispielsweise um Datenbanken.
Der Artikel von \citeauthor{goldberg91} \cite{goldberg91} ist auch ganz interessant.
Zum Schluss noch eine online Quelle \cite{wave} und eine lange URL \cite{long}, die im \nameref{bib} hoffentlich ordentlich auf mehrere Zeilen aufgeteilt wird.

% \section{Ligaturtest}

% \begin{description}
%   \item[Ligaturen erlaubt]{Affe, flink, offiziell}
%   \item[Ligaturen verboten]{Kaufleute, Ablauffolge, Stoffjacke}
% \end{description}
